\documentclass[12pt]{article}
\bibliographystyle{plain}

\usepackage[sc]{mathpazo}
\usepackage[utf8]{inputenc}

\let\[\equation
\let\]\endequation

\begin{document}

This document sketches out the calculation of autocorrelation functions.  


First, suppose we have calculate the left eigenvectors and eigenvalues:

$$T^T \phi_i = \lambda \phi_i$$

Suppose that $\pi$ is the equilibrium population.  Then, we can normalize the eigenvectors such that:

$$\phi_i^T \pi^{-1} \phi_j = \delta_{ij}$$

Above, we denote $\pi^{-1}$ to be a diagonal matrix with elements $\pi_i^{-1}$.

The autocorrelation function of the observable $f_i$ can be denoted:

$$E(f(z_t) f(z_0)) = \sum_{i,j} f_i P(z_0 = i) f_j P(z_t = j | z_0 = i) = \sum_{i,j} f_i  f_j \pi_i T_{ij} = $$

We know that

$$T_{ab}(t) = \sum_k \lambda_k(t) (\psi_k)_a (\phi_k)_b = \sum_k \lambda_k(t) (\pi_a)^{-1} (\phi_k)_a (\phi_k)_b$$

Thus,

$$E(f(z_t) f(z_0)) = \sum_{i,j,k} f_i f_j \lambda_k(t) (\phi_k)_i (\phi_k)_j = \sum_k \lambda_k(t) s_k^2$$

Where 

$$s_k = \sum_i f_i (\phi_k)_i$$

Finally, note that $\lambda_i(\infty) = \delta_{i0}$, so the long-timescale behavior is simply:

$$E(f(z_\infty) f(z_0)) = s_0^2$$

For most applications, one is interested in the zero-centered ACF, so we simply skip the $k = 0$ term in the summation.

\end{document}